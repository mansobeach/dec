%
% Elecnor Deimos
%
% Data Exchange Component
%
% Borja Lopez Fernandez (BOLF)
%
% DEC Install information
%

\documentclass[dec_sum_main.tex]{subfiles}
 
\begin{document}

\section{Install Environment}

This section covers the install of \textit{Ruby} interpreter and the \textit{gem} dependencies needed by DEC SW. It is recommended that the \textit{Ruby} interpreter is done locally for the Linux user who shall execute the DEC SW.

\par
\noindent

\subsection{Install Ruby}
 
The \textit{Ruby} interpreter can be obtained and installed in many different ways ; this manual describes how to install it local to the user which shall execute the DEC SW.

\par
\noindent
The install is based on \href{https://rvm.io/}{RVM}, which is a command-line tool which allows you to easily install, manage, and work with multiple ruby environments from interpreters to sets of gems. \newline

\par
\inlinecode{bash}{ \$> \\curl -sSL https://get.rvm.io | bash -s stable --ruby} \newline
\par 
\noindent
Alternatively to the latest ruby stable installation, the following command installs a specific version of ruby interpreter ; the version specified below corresponds to the one used for the unit tests and it is required as a mandatory precondition at installation time according to the gem file definition.\newline

\par 
\noindent
Different versions can be locally installed and select the one of choice at any time. However the following ruby version is recommended:

\inlinecode{bash}{ \$> rvm install ruby-3.0} \newline

\inlinecode{bash}{ \$> rvm --default use 3.0} \newline


\section{Install DEC}

In order to install DEC SW and the gems required, execute the following command in the shell at the directory in which the DEC gem file is placed:
\par
\inlinecode{bash}{ \$> gem install dec_stable.gem } \newline
\par

\par
\par
\noindent
It is usually recommended to perform the installation of every DEC node with a dedicated installer, which already carries the desired configuration according to the defined interfaces and the desired behaviour ; this approach makes installations and configuration into the target environment almost instantaneously (e.g. maintenance in Operations). Every DEC node configuration can be kept under configuration control in order to build the dedicated installers.\newline

\noindent
The DEC installer naming file is generated to avoid ambiguity and bring information regarding the node they apply to. Below there are some few examples to illustrate the naming conventions to easily identify the installation kit for a given DEC node configuration.\newline
\par

\inlinecode{bash}{ \$> gem install dec-1.0.32_s2_dec@s2boa-cloudferro.gem } \newline

\inlinecode{bash}{ \$> gem install dec-1.0.33_s2_push_lisboa@e2espm-inputhub.gem } \newline

\inlinecode{bash}{ \$> gem install dec-1.0.33_unit_tests@localhost.gem } \newline

\noindent
Also the DEC installer can be customised to carry or avoid specific SW items, such as the testing tools (i.e. unit tests, interface tests, etc). The installation kit referred below as example has been customised for some Sentinel-2 project to carry the test tools, the OData tools and make usage of postgresql to persist the operations performed \newline

\inlinecode{bash}{ \$> gem install dec-1.0.33_s2_test_pg_odata@localhost.gem } \newline

\noindent
As a very quick summary, users of the SW are encouraged to delegate the creation of the configuration by providing the requirements / interface documents to build dedicated installation packages for every node. 
\par


\section{Uninstall DEC}

In order to uninstall DEC SW, execute the following command in the shell:
\par 
\inlinecode{bash}{ \$> gem uninstall dec }
\par 

\begin{verbatim}
Remove executables:
decValidateConfig, decCheckConfig, decCheckSent, decConfigInterface2DB, decDeliverFiles, decGetFiles4Transfer, decGetFromInterface, decListener, decManageDB, decNotify2Interface, decSend2Interface, decSmokeTests, decStats, decUnitTests, decUnitTests_IERS, decUnitTests_ncftpput, decUnitTests_mail

in addition to the gem? [Yn]
\end{verbatim}

Press 'Y' key to remove the executables as well

\label{COTS}
\section{COTS Required}

This section enumerates the COTS which are used by DEC SW for exchange of file by some network protocol implementation, or file transformations associated to those exchanges. \par\noindent


\par
\noindent
This manual, which currently address component level information, does not address how to provision these COTS ; they can be obtained naturally with most OS distribution, downloaded with its native package manager, or manually installed. However it is noted that DEC containerized execution environments definition (IaaS) which already resolve every SW COTS dependencies out of the box are available, please do not hesitate in requesting information.

\subsection{Databases}
This section enumerates the different databases which can be used by DEC SW. Only \textit{sqlite3} is \textit{mandatory} to allow the execution of the entire set of \textit{unit tests}. Below the different databases that have used at some deployment. It is recalled that it is possible to execute the DEC SW without any database by usage of flag \textit{"--nodb"}.
\par
\begin{itemize}
	\item \href{https://www.sqlite.org}{\textit{sqlite3}} : is an in-process library that implements a self-contained, serverless, zero-configuration, transactional SQL database engine
	\item \href{https://www.postgresql.org/}{\textit{PostgreSQL}} : object-relational database system with a strong reputation for reliability, feature robustness, and performance
	\item \href{https://www.mysql.com/}{\textit{MySQL}} : a high performance, scalable database management system
\end{itemize}

\subsection{Network Tools}
This section enumerates the network tools which can be used by DEC SW.
\par
\begin{itemize}
	\item \href{https://www.ncftp.com}{\textit{ncftp}} : application programs implementing the File Transfer Protocol (FTP)
	\item \href{https://www.openssh.com}{\textit{sftp}} : application programs implementing the Secure File Transfer Protocol (SFTP)
	\item \href{https://curl.haxx.se}{\textit{curl}} : command line tool and library for transferring data with URLs (WebDAV)
\end{itemize}

\subsection{File Compression Tools}
This section enumerates the file compression tools which can be used by DEC SW.
\par
\begin{itemize}
	\item \href{https://www.7-zip.org/}{\textit{7-zip}} : is a file archiver with a high compression ratio ; name of the package can be \textit{"p7zip"}
	\item \href{http://infozip.sourceforge.net}{\textit{zip} / \textit{unzip}} : provide free, portable, high-quality versions of the \textit{Zip} and \textit{UnZip} compressor-archiver utilities
	\item \href{https://www.gzip.org}{\textit{gzip}} : The \textit{gzip} reduces the size of the named files using Lempel–Ziv coding (LZ77)
	\item \href{http://man7.org/linux/man-pages/man1/compress.1p.html}{\textit{compress}} : The \textit{compress} utility reduces the size of the named files by using adaptive Lempel-Ziv coding algorithm
\end{itemize}

\subsection{File Transformation Tools}
This section enumerates the file transformation tools which can be used by DEC SW.

\begin{itemize}
	\item \href{http://xmlsoft.org/xmllint.html}{\textit{xmllint}} : is a command line XML parser which is part of the \href{http://xmlsoft.org}{\textit{libxml2}} and libxml2-utils packages.
    \item \href{https://stedolan.github.io/jq/}{\textit{jq}} : jq is a lightweight and flexible command-line JSON processor.	
\end{itemize}

\section{Installation Verification}
\subsection{Verification with Unit Tests}
This section describes how to execute the \textit{unit tests}, which have been designed to be transparent and harmless in front of the potential different execution environments (i.e. development, integration, production), being their execution a simple and effective manner to verify the correct installation of the DEC SW. \newline
\par
The prerequisites to be able to successfully execute the \textit{unit tests} are :
\begin{itemize}
	\item an OS user \textit{dectest}
	\item SFTP server running on \textit{localhost} allowing login to \textit{dectest} using the SSH keys
	\item FTP server running on \textit{localhost} allowing login to \textit{dectest} using the \textit{password} \textit{dectest} \newline
\end{itemize} 
\par
\inlinecode{bash}{ \$> decUnitTests } \newline

\noindent
The results of the unit test should show no failures neither errors ; the execution time in a 2.66 GHz Intel Core 2 Duo is about 10 minutes approximately.
\lstset{escapeinside={<@}{@>}}
\begin{lstlisting}
.
Finished in 590.39725 seconds.
<@\textcolor{green}{-----------------------------------------------------------------------------------------------------------------------------------------------------------}@>
17 tests, 112 assertions, 0 failures, 0 errors, 0 pendings, 0 omissions, 0 notifications
100% passed
<@\textcolor{green}{-----------------------------------------------------------------------------------------------------------------------------------------------------------}@>
0.03 tests/s, 0.19 assertions/s
\end{lstlisting}

\subsection{Verification with Operational Interface}
This section describes how to verify the correct installation of DEC SW by execution of a test with the IERS \textit{operational} service, for which Internet connectivity is required for the FTP protocol. Note that it is not possible to ensure the connectivity availability by such service and sometimes test may fail by reply of \textit{530 connect failed: Address already in use. No response from server.} \newline

\inlinecode{bash}{ \$> decUnitTests_IERS } \newline

\begin{lstlisting}
.
Finished in 54.38868 seconds.
<@\textcolor{green}{-----------------------------------------------------------------------------------------------------------------------------------------------------------}@>
2 tests, 25 assertions, 0 failures, 0 errors, 0 pendings, 0 omissions, 0 notifications
100% passed
<@\textcolor{green}{-----------------------------------------------------------------------------------------------------------------------------------------------------------}@>
0.02 tests/s, 0.33 assertions/s 
\end{lstlisting}

\noindent
\newline
In case of deployment without any database, the tests can be restricted to the ones which make usage of the \textit{"--nodb"} execution option:

\inlinecode{bash}{ \$> decUnitTests_IERS -n test_decGetFromInterface_NODB} \newline

\begin{lstlisting}
.
Finished in 54.38868 seconds.
<@\textcolor{green}{-----------------------------------------------------------------------------------------------------------------------------------------------------------}@>
1 tests, 18 assertions, 0 failures, 0 errors, 0 pendings, 0 omissions, 0 notifications
100% passed
<@\textcolor{green}{-----------------------------------------------------------------------------------------------------------------------------------------------------------}@>
0.02 tests/s, 0.23 assertions/s
\end{lstlisting}

\end{document}

